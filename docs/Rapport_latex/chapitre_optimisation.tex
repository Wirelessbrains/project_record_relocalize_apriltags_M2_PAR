\newpage
\section{Otmizacao e trajetoria }


Apos um visualizaçao em que um camera se move em uma trajetoria que se 
paraece com incosaedro, durante a visualizaçao foi garantido que a todo
momento a camera ve pelo menos duas tags;

A primeira etapa a realizar é fazer mapaemento do nosso ambiente. Sendo 
assim, definir a primeira tag referencia o calculadas tranformaçoes 
relativas para as tags subsequentes. dado que eu sei onde todas as tags 
estao localizada posiçao e orientaçao, eu posso localizar o meu 
robo no ambiente.

A problemantica é que o robo nao consegue ver todas as tags ao mesmo tempo 
e também, dado que o robo ve as tags em sequencia pequenos erros de 
detecçao da posicao do robo sao acumulados a medida em que o robo se
move

Logo o mapaemento deve ser em caideia fazendo as transformada de referencial
a partir de uma tag referencia, ou a tag mundo.

A camera é colocada como pose de referencia e recebe a pose relativa da
tagA, primeira tag vista, ao mesmo tempo, é possivel ver a tagB, Assim
atraves destas poses é possivel cacula a pose da tag B em relaçao a tagA:

\begin{equation}
    T_{A \to B} = T_{Cam \to A}^{-1} * T_{Cam \to B}
\end{equation}

\begin{itemize}
    \item $T_{A \to B} = (R_A, \bm{t}_A)$: Pose da Tag B em relaçao a A.
    \item $T_{Cam \to A} = (R_A, \bm{t}_A)$: Pose da Tag A vista pela Câmera.
    \item $T_{Cam \to B} = (R_B, \bm{t}_B)$: Pose da Tag B vista pela Câmera.
    \item $R$: Matriz de rotaçao
    \item $t$: Vetor de posiçao
\end{itemize}

A tranformaçao de inveresa é caculada da seguinte forma:

\begin{equation}
    T^{-1} = (R^T, -R^T \bm{t})
\end{equation}

dada que o as orientaçoes das tags sao dadas em quartenion, é feita a conversao 
para matrizes de rotacao:


\begin{equation}
R = \begin{bmatrix}
q_0^2 + q_1^2 - q_2^2 - q_3^2 & 2(q_1q_2 - q_0q_3) & 2(q_0q_2 + q_1q_3) \\
2(q_1q_2 + q_0q_3) & q_0^2 - q_1^2 + q_2^2 - q_3^2 & 2(q_2q_3 - q_0q_1) \\
2(q_1q_3 - q_0q_2) & 2(q_0q_1 + q_2q_3) & q_0^2 - q_1^2 - q_2^2 + q_3^2
\end{bmatrix}
\end{equation}

Assim é possivel calcular a rotaçao relativa($R_{rel}$):

\begin{equation}                                                          
R_{rel} = R_A^T R_B                                                   
\end{equation}

Dado que a detecao da tag também nos fornece a posiçao, assim também
é possivel calcular: a translaçao relativa ($\bm{t}_{rel}$):

\begin{equation}                                                          
     \bm{t}_{rel} = R_A^T (\bm{t}_B - \bm{t}_A)                            
\end{equation}

Assim

\begin{equation}
    T_{A \to B} = (R_{rel}, \bm{t}_{rel})
\end{equation}

também é possivel realisar o calculo atravez de matrizes de transformação homogenea:

                                                          
                                                             
                   
\begin{equation}
\begin{bmatrix} 
R_{rel} & \bm{t}_{rel} \\ 
\mathbf{0}_{1 \times 3} & 1 
\end{bmatrix} = 
\begin{bmatrix} 
R_A & \bm{t}_A \\ 
\mathbf{0}_{1 \times 3} & 1 
\end{bmatrix}^{-1} 
\begin{bmatrix} 
R_B & \bm{t}_B \\ 
\mathbf{0}_{1 \times 3} & 1 
\end{bmatrix}
\end{equation}







\subsubsection{Subsubsection}
